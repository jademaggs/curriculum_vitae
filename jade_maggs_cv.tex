\documentclass[10pt,a4paper]{article}
\usepackage[utf8]{inputenc}
\usepackage{amsmath}
\usepackage{amsfonts}
\usepackage{amssymb}
\usepackage{fancyhdr} % headers and footers
\usepackage{graphicx}
\usepackage[T1]{fontenc}
\usepackage[left=1.5cm, right=1.5cm, top=1.5cm, bottom=1.5cm]{geometry}
\usepackage[svgnames]{xcolor} %To access named colors below
\usepackage{hyperref}
\hypersetup{colorlinks,citecolor=DeepPink4,linkcolor=DarkRed, urlcolor=DarkBlue}
\usepackage{lastpage}
\usepackage{enumitem} % to itemise with reduced spacing
\usepackage{times}       % Times New Roman
\usepackage{lastpage} % for defining last page
\usepackage{hanging} % hanging indent for references
\usepackage{hyperref} % for hyperlinks 

\pagestyle{fancy} % Adds header bar 

\author{Jade Maggs}
\title{Jade Maggs Resume}

% ------------------------------------------------------------------
% Set up headers and footers
% ------------------------------------------------------------------
\fancyhf{} % sets both header and footer to nothing
\renewcommand{\headrulewidth}{0pt}

%\chead{CURRICULUM VITAE}
\lfoot{}
\cfoot{}
\rfoot[\footnotesize{\sffamily{}}]{page \thepage\ of \pageref{LastPage}}

% ------------------------------------------------------------------
% Document start
% ------------------------------------------------------------------
\begin{document}
	
\noindent
{\huge \textbf{Dr Jade Quinton Maggs}}\\
\\
%{\small Citizenship: New Zealand, South Africa\\}
{\small Auckland, New Zealand | +6421 056 0873 | \href{mailto:jademaggs@gmail.com}{jademaggs@gmail.com} |
\href{https://www.linkedin.com/in/jade-maggs-phd-5a091729/}{LinkedIn} |
			\href{https://www.researchgate.net/profile/Jade_Maggs}{ResearchGate} | 
			\href{https://www.scopus.com/authid/detail.uri?authorId=55250227800}{Scopus \textit{h}-index: 12} |
			\href{https://github.com/jademaggs?tab=repositories}{GitHub}}\\
\hrule
\vspace{8pt}
\noindent
I am a Fisheries Scientist with 15 years of experience and a strong track record in collaborative research. I currently work in the Wild Fisheries Flagship Programme at Earth Sciences New Zealand (formerly NIWA), where I am the Group Manager for the Modelling \& Recreational Fisheries team. I lead national boat ramp creel surveys of the recreational fishery, with a focus on inshore fin-fish and rock lobster fisheries. In 2017, I earned a PhD in Fisheries Science from Rhodes University in South Africa. My expertise also includes data management and GIS, and I have lectured part-time at postgraduate level on fishery stock assessment and R programming. \\
%Previously, I held a Senior Analyst role with Fisheries New Zealand within the Fisheries Management department. Before that I held a Scientist role with the Oceanographic Research Institute in South Africa with a focus on monitoring marine recreational fisheries and marine protected areas. I have published quantitative fisheries research in international peer-reviewed journals. In 2017, I was awarded a PhD in fisheries science from Rhodes University in South Africa. I have also lectured part-time at postgraduate level on fishery stock assessment and R programming. I am a highly organised individual and I excel at leading change. 
\hrule
\vspace{8pt}
\noindent

% ------------------------------------------------------------------
% EMPLOYMENT HISTORY
% ------------------------------------------------------------------

\section*{Professional Experience}



\textbf{\href{https://earthsciences.nz/}{EARTH SCIENCES NEW ZEALAND}} \text{(formerly NIWA)} \\
\text{Fisheries Scientist | Group Manager} \\
\textit{Jan 2022 -- present}
\begin{itemize}[itemsep=2pt, parsep=0pt]
  \item Group manager - Modelling and Recreational Fisheries 
  \item Project leader for national boat ramp creel surveys of the recreational fishery and other projects
  \item Fisheries modelling (GLM, CPUE standardisation, harvest estimation, recreational release mortality)
  \item Member of the Marine Amateur Fisheries and Inshore Fisheries Assessment working groups
  \item Using R, PostgreSQL, Linux, QGIS, GitLab
  \item Leading implementation of electronic data acquisition and streamlining workflows
  \item Selected for in-house Leadership Development Programme
\end{itemize}



\noindent \text{Group Manager – Fisheries Data Services} \\
\textit{Apr 2020 -- Dec 2021}
\begin{itemize}[itemsep=2pt, parsep=0pt]
  \item Managed all fisheries data from research and observer programmes for NIWA and the Ministry for Primary Industries
  \item Managed commercial rock lobster catch sampling project
  \item Led development of data loading software and PostgreSQL databases
  \item Managed a team of six staff
  \item Used R, PostgreSQL, Linux, QGIS, GitLab
  \item Delivered efficiencies and significant improvements across multiple workflows\\
\end{itemize}



\noindent \textbf{\href{https://www.mpi.govt.nz}{MINISTRY FOR PRIMARY INDUSTRIES}} \\
\text{Senior Analyst – Fisheries Management} \\
\textit{Jan 2019 -- Mar 2020}
\begin{itemize}[itemsep=2pt, parsep=0pt]
  \item Contributed to the development of ecosystem-based fisheries management approaches
  \item Member of the Marine Protected Area (MPA) Science Advisory Group
  \item Engaged with stakeholders on proposed MPA networks, including those affecting commercial rock lobster fisheries\\
\end{itemize}



\noindent \textbf{\href{https://saambr.org.za/oceanographic-research-institute-ori/}{OCEANOGRAPHIC RESEARCH INSITITUTE}} \text{(South Africa)} \\
\text{Scientist} \textit{Jun 2018 -- Dec 2018} \\
\text{Assistant Scientist} \textit{Aug 2011 -- May 2018} \\
\text{Senior Scientific Technician}  \textit{Jul 2008 -- Jul 2011}\\
\text{Scientific Technician} \textit{May 2006 -- Jun 2008}
\begin{itemize}[itemsep=2pt, parsep=0pt]
  \item Led recreational fisheries surveys (shore- and boat-based)
  \item Fisheries modelling, including GLMs, CPUE standardisation, and fish movement analysis
  \item Field surveys to monitored relative abundance, movement, and community composition of fishes
  \item Delivered postgraduate lectures on fisheries stock assessment and R programming
  \item Performed biological sampling of commercial fish species
  \item Developed and administered fisheries monitoring databases using PostgreSQL
  \item Member of the Linefish Scientific Working Group
  \item Managed a team of three staff and a research aquarium laboratory\\
\end{itemize}



\noindent \textbf{\href{https://saambr.org.za/ushaka-sea-world//}{uSHAKA SEA WORLD}} \text{(South Africa)} \\
\text{Aquarist II} \textit{Feb 2006 - Apr 2006}\\
\text{Aquarist I} \textit{ Feb 2005 - Jan 2006}\\
\text{Commercial Diver} \textit{Jul 2004 - Jan 2005}
 \begin{itemize}[itemsep=2pt, parsep=0pt]
   \item Showed an affinity for animal husbandry - promoted from Diver to Aquarist
   \item Maintained aquarium exhibits and cultures laboratory
   \item Developed fish breeding and coral propagation facilities
\end{itemize}


\hrule	
\vspace{6pt}
\noindent


% ------------------------------------------------------------------
% ACADEMIC QUALIFICATIONS
% ------------------------------------------------------------------

\noindent
\section*{Academic Qualifications}
\begin{itemize}[itemsep=2pt, parsep=0pt]
   \setlength\itemsep{0.05em}
   \item PhD (Fisheries Science) | Rhodes University, 2017
      \begin{itemize}[itemsep=2pt, parsep=0pt]
	\setlength\itemsep{0.05em}
	\item Thesis title: Movement of coastal fishery species in southern Africa: research trends, characterisation of behaviours and a case study on fishery implications. Supervisor: Prof. Paul Cowley.
      \end{itemize}
		\item MSc (Biology) | University of KwaZulu-Natal, 2011
		\begin{itemize}[itemsep=2pt, parsep=0pt]
		\setlength\itemsep{0.05em}
			\item Thesis title: Fish surveys in exploited and protected areas of the Pondoland Marine Protected Area with consideration of the impact of the MPA on coastal fisheries. Supervisors: Dr Bruce Mann \& Prof. Paul Cowley.
		\end{itemize}
		\item BSc (Zoology, specialising in environmental management) | University of South Africa, 2009 
	\end{itemize}

\hrule
\vspace{6pt}

\noindent
\section*{Lecturing Experience}
	\begin{itemize}[itemsep=2pt, parsep=0pt]
		\setlength\itemsep{0.05em}
		\item Fishery stock assessment (University of KwaZulu-Natal, 2012-2018) - Postgraduate
		\item Introduction to R Programming (2016, 2017) - Postgraduate
	\end{itemize}

\hrule
\vspace{6pt}
\noindent
\section*{Technical Skills}
	\begin{itemize}[itemsep=2pt, parsep=0pt]
		\setlength\itemsep{0.05em}
		\item \textbf{Software}: R programming | GIT | Quantum GIS | Ubuntu Linux | Bash | \LaTeX\
		\item \textbf{Statistics}: Maximum likelihood | Generalised linear modelling
		\item \textbf{Data management}: PostgreSQL 
		%\item \textbf{Communication}: Peer-reviewed journal publication | postgraduate lecturing | oral presentations at conferences
		\item \textbf{Field}: Class IV Commercial Diver (563 hours) | Class IV Commercial Dive Supervisor 
		(313 hours) | Commercial Boat Skipper | Level 2 First Aid | STCW - PST, PSSR | servicing underwater environmental 
		monitoring equipment | fish handling and tagging | specimen collection (biological/environmental)
	\end{itemize}	

\hrule
\vspace{6pt}
\noindent
\section*{Technical Courses Attended}
	\begin{itemize}[itemsep=2pt, parsep=0pt]
		\setlength\itemsep{0.05em}
		\item Zonation decision support software training (NIWA, Sep 2019)
		\item Ecofish stock assessment interpretation (Dept. of Agriculture, Forestry and Fisheries, Apr 2017)
		\item Introduction to \LaTeX{} (University of KwaZulu-Natal, Jun 2016)
		\item Introduction to Fisheries Genetics (University of Stellenbosch, Mar 2016)
		\item Species distribution modelling (University of Stellenbosch, Feb 2013) 
		\item Ecological modelling (IRD/University of Cape Town, Jun 2012)
		\item Statistical modelling Using R (Troms\o\ University, Sep 2011)
		\item Fish stock assessment (FAO/SWIOFP - Kenya, Sep-Oct 2010)
	\end{itemize}
\hrule
\vspace{6pt}
\noindent
%\section*{Ship Time}
%	\begin{itemize}[itemsep=2pt, parsep=0pt]
%		\setlength\itemsep{0.05em}
%		\item RV Kaharoa, Cook Strait, Jul/Aug 2021 - 2 days - KAH2105 acoustic snapshot, trawling
%		\item FRV Africana, Natal Bight, Sep 2010 - 5 days - retrieve and deploy ADCPs
%		\item TB Davies, Thukela Banks, Apr 2008 - 3 days benthic grab sampling
%		\item Ocean Stroom, Thukela Banks, Aug 2005 - 1 day - relocate two bull sharks
%		\item Ocean Stroom, Illovo Banks, Aug 2004 - 1 day - specimen collection
%	\end{itemize}
%\hrule
%\vspace{6pt}
\noindent
\section*{Referees}


%\textbf{Dr Bradley Moore} | XX Scientist - Marine XX | Earth Sciences NZ, Hobart\\
%Contact: +6 | \href{mailto: XX@XX}{XX@XX}\\
%\\
\textbf{Dr Darren Parsons} | Principal Scientist - Marine Ecology | Earth Sciences NZ, Auckland\\
Contact: +64 21 170 1724 | \href{mailto: Darren.Parsons@niwa.co.nz}{Darren.Parsons@niwa.co.nz}\\
\\
%\textbf{Mr Jonathan Moores} | Regional Manager | NIWA, Auckland (current manager)\\
%Contact:+64 27 233 7871 | \href{mailto:Jonathan.Moores@niwa.co.nz}{Jonathan.Moores@niwa.co.nz}\\
%\\
%\textbf{Dr Alison MacDiarmid} | Regional Manager | NIWA, Wellington\\
%Contact: +64 274 033 468 | \href{mailto:Alison.MacDiarmid@niwa.co.nz}{Alison.MacDiarmid@niwa.co.nz}\\
%\\
\textbf{Dr Richard Saunders} | Impact \& Engagement Officer | Integrated Marine Observing System, Hobart\\
Contact: +61 483 350 831 | \href{mailto:Richard.Saunders@utas.edu.au}{Richard.Saunders@utas.edu.au}\\
\\
%\noindent
%\textbf{Dr Bruce Mann} | Senior Scientist | Oceanographic Research Institute, South Africa\\
%Contact: +2772 356 6629 | \href{mailto:bruce@ori.org.za}{bruce@ori.org.za}\\
%\\
%\noindent
%\textbf{Dr Paul de Bruyn} | Science Manager | Indian Ocean Tuna Commission, Seychelles\\
%Contact: +248 422 5494 | \href{mailto:paul.debruyn@fao.org}{paul.debruyn@fao.org}\\
%Contact: +34 671 57 31 13 | \href{mailto:paul.debruyn@fao.org}{paul.debruyn@fao.org}\\

\hrule
\newpage
\vspace{6pt}
\noindent
\section*{Publications}

Career summary: 24 peer-reviewed journal articles, 21 working group presentations, 1 book section, 2 theses, 2 conference proceedings, 81 reports, 22 conference presentations, 2 posters, 16 popular articles\\ 
\\
\textit{\textbf{Selected publications - chronologically}}\\

\begin{hangparas}{1.5em}{1}

\underline{Maggs JQ}, Holdsworth JC, Curtis S, Evans OE, Bodie CEH, Madden BK. \textit{in prep}. Estimating release mortality in New Zealand’s largest recreational fishery. \textit{Fisheries Management and Ecology}.

Martinez S, Bernard ATF, Speed CW, Mann BQ, Olbers JM, \underline{ Maggs JQ}, Floros C, Meekan MG, Yon A. 2024. Elasmobranch assemblage structure on protected high-latitude coral reefs of southeast Africa. \textit{Marine Ecology Progress Series} 749: 87-107. \url{https://doi.org/10.3354/meps14717}

Smallwood CB, Ryan KL, Flanagan EA,\underline{ Maggs JQ}, Ochwada-Doyle FA, Tracey SR. 2024. Spiny lobster recreational fisheries in Australia and New Zealand: An overview of regulations, monitoring, assessment and management. \textit{Fisheries Research} 280: 107149. \url{https://doi.org/10.1016/j.fishres.2024.107149}

Goetze J, Heithaus MR, MacNeil MA, [...], \underline{Maggs JQ}, [...].2024. Directed conservation of the world’s reef sharks and rays. \textit{Nature Ecology \& Evolution} 8: 1118-1128. \url{https://doi.org/10.1038/s41559-024-02386-9}

\underline{Maggs JQ},  Evans O, Holdsworth JC, Lumley T, Hartill BW. 2024. Post-release mortality of line-caught snapper \textit{Chrysophrys auratus} depends on hook site and capture depth. \textit{Fisheries Management and Ecology} 31(4): e12702. \url{https://doi.org/10.1111/fme.12702}

Johnson KS, \underline{Maggs JQ}, Taylor R, Evans OE, Armiger H. 2024. Estimation of recreational harvest of red and packhorse rock lobster in the CRA 1 area (2022-23). \textit{New Zealand Fisheries Assessment Report} 2024/54. 21 p. \url{https://fs.fish.govt.nz/Page.aspx?pk=113&dk=25792}

\underline{Maggs JQ}, Evans OE, Taylor R, Armiger H, Marsh C, Hartill BW. 2024. Monitoring of recreational harvest of red rock lobster \textit{Jasus edwardsii} in CRA 2. \textit{New Zealand Fisheries Assessment Report}  2024/52. 41 p. \url{https://fs.fish.govt.nz/Page.aspx?pk=113&dk=25801}

\underline{Maggs JQ}, Armiger H, Evans O, Taylor R, Davey N, Payne G, Miller A, Spong K, Parkinson D, Bian R, Hartill BW. 2024. Trends in recreational boat effort and harvest from 2004-05 to 2022-23. \textit{New Zealand Fisheries Assessment Report} 2024/53. 60 p. \url{https://fs.fish.govt.nz/Page.aspx?pk=113&dk=25806}

Davey NK, Johnson KS, \underline{Maggs JQ}. 2024. Mean weight estimates for recreational fisheries in 2022-23. \textit{New Zealand Fisheries Assessment Report} 2024/28. 39 p. \url{https://fs.fish.govt.nz/Page.aspx?pk=113&dk=25758}

\underline{Maggs JQ}, Parsons D. 2023. Design for catch sampling programme to estimate the age structure of New Zealand h\a=apuku (\textit{Polyprion oxygeneios}). \textit{New Zealand Fisheries Assessment Report 2023/34}. 30 p. \url{https://fs.fish.govt.nz/Page.aspx?pk=113&dk=25522}

\underline{Maggs JQ}, Davey NK, Hartill BW. 2023. Monitoring the Kaik\=oura and Motunau boat-based marine amateur fisheries for blue cod, sea perch and rock lobster, 2021-22. \textit{New Zealand Fisheries Assessment Report 2023/33}. 53 p \url{https://fs.fish.govt.nz/Page.aspx?pk=113&dk=25548}

Simpfendorfer CA, Heithaus MR, Heupel MR, [...], \underline{Maggs JQ}, [...], Chapman D. 2023. Widespread diversity deficits of coral reef sharks and rays. \textit{Science} 380: 1155-1160. \url{https://doi.org/10.1126/science.ade4884}

Mann BQ, \underline{Maggs JQ}, Porter SN, Dalton WN. 2022. Monitoring the effects of spatial protection on the reef fish communities of the Pondoland Marine Protected Area, Eastern Cape Province, South Africa. \textit{African Journal of Marine Science} 44(2): 1-17. \url{https://doi.org/10.2989/1814232X.2022.2047782}

Dames V, Bernard ATF, Floros C, Mann B, Speed C, \underline{Maggs J}, Laing S, Meekan M, Olbers J. 2020. Zonation and reef size significantly influence fish population structure in an established marine protected area, iSimangaliso Wetland Park, South Africa. \textit{Ocean \& Coastal Management} 185: 105040.\url{https://doi.org/10.1016/j.ocecoaman.2019.105040}

MacNeil MA, Chapman D, Heupel M, [...], \underline{Maggs JQ}. 2020. Global status and conservation potential of reef sharks. \textit{Nature} 583: 801-806. \url{https://doi.org/10.1038/s41586-020-2519-y}

Murray TS, Cowley PD, Mann BQ, \underline{Maggs JQ}, Gouws G. 2019. Movement patterns of an endemic South African sparid, the black musselcracker \textit{Cymatoceps nasutus}, determined using mark-recapture methods. \textit{African Journal of Marine Science} 41(1): 71-81. \url{http://dx.doi.org/10.2989/1814232X.2019.1574238}

\underline{Maggs JQ}, Cowley PD, Porter SN, Childs AR. 2019. Should I stay or should I go? Intra-population variability in movement behaviour of wide-ranging and resident coastal fishes. \textit{Marine Ecology Progress Series} 619: 111-124. \url{https://doi.org/10.3354/meps12953}

Heyns-Veale ER, Bernard ATF, G\"otz A, Mann BQ, \underline{Maggs JQ}, Smith MKS. 2019. Community-wide effects of protection reveal insights into marine protected area effectiveness for reef fish. \textit{Marine Ecology Progress Series} 620: 99-117. \url{http://dx.doi.org/10.3354/meps12970}

\underline{Maggs J}, Parker D, Kerwath S, Attwood C, da Silva C, Winker H. 2017. The 2017 assessment of slinger (\textit{Chrysoblephus puniceus}) for the South African linefishery. Report of the Linefish Scientific Working Group, LSWG No. 01, 2017: 27 pp.

\underline{Maggs JQ}, Mann BQ, Potts WM, Dunlop SW. 2015. Traditional management strategies fail to arrest a decline in the CPUE of an iconic marine recreational fishery species with evidence of hyperstability. \textit{Fisheries Management and Ecology} 23(3-4): 187-199. \url{http://dx.doi.org/10.1111/fme.12125}

\underline{Maggs JQ}, Cowley PD. 2016. Nine decades of fish movement research in southern Africa: a synthesis of research and findings from 1928 to 2014. \textit{Reviews in Fish Biology and Fisheries} 26(3): 287-302. \url{https://doi.org/10.1007/s11160-016-9425-2}

Mann BQ, Winker H, \underline{Maggs JQ}, Porter S. 2016. Monitoring the recovery of a previously exploited surf-zone fish community in the St Lucia Marine Reserve, South Africa, using a no-take sanctuary area as a benchmark. \textit{African Journal of Marine Science} 38(3): 423-441. \url{https://doi.org/10.2989/1814232X.2016.1224779}

Mann BQ, \underline{Maggs JQ}, Khumalo MC, Khumalo D, Parak O, Wood J, Bachoo S. 2015. KwaZulu-Natal Boat Launch Site Monitoring System: A novel approach for improved management of small vessels in the coastal zone. \textit{Ocean and Coastal Management} 104(2015): 57-64. \url{https://doi.org/10.1016/j.ocecoaman.2014.12.003}

Dunlop SW, Mann BQ, Cowley PD, Murray TS, \underline{Maggs JQ}. 2015. Movement patterns of \textit{Lichia amia} (Teleostei: Carangidae): results from a long-term cooperative tagging project in South Africa. \textit{African Journal of Zoology} 50(3): 249-257. \url{http://dx.doi.org/10.1080/15627020.2015.1058724}

\underline{Maggs JQ}, Mann BQ, Cowley PD. 2013. The contribution of a large no-take area to the management of four heavily exploited near-shore linefish species along the south-east African coast. \textit{Fisheries Research} 144(2013): 38-47. \url{https://doi.org/10.1016/j.fishres.2012.10.003}

\underline{Maggs JQ}, Mann BQ, Cowley PD. 2013. Reef fish display station-keeping and ranging behaviour in the Pondoland Marine Protected Area on the east coast of South Africa. \textit{African Journal of Marine Science} 35(2): 183-193. \url{https://doi.org/10.2989/1814232X.2013.798152}

Floros C, Schleyer MH, \underline{Maggs JQ}. 2013. Fish as indicators of diving and fishing pressure on high-latitude coral reefs. \textit{Ocean and Coastal Management} 84: 130-139. \url{https://doi.org/10.1016/j.ocecoaman.2013.08.005}

\underline{Maggs JQ}, Mann BQ, van der Elst RP. 2012. Long-term trends in the management of the recreational shore-fishery for elf (\textit{Pomatomus saltatrix}: Pomatomidae) in KwaZulu-Natal, South Africa. \textit{African Journal of Marine Science} 34(3): 401-410. \url{http://dx.doi.org/10.2989/1814232X.2012.709959}

Mann BQ, Pradervand, P, \underline{Maggs JQ}, Wintner S. 2012. A characterisation of the paddle-ski fishery in KwaZulu-Natal, South Africa. \textit{African Journal of Marine Science} 34(1): 119-130. \url{https://doi.org/10.2989/1814232X.2012.675120}

Floros C, Schleyer MH, \underline{Maggs JQ}, Celliers L. 2012. Baseline assessment of high-latitude coral reef fish communities in southern Africa. \textit{African Journal of Marine Science} 34(1): 55-69. \url{http://dx.doi.org/10.2989/1814232X.2012.673284}

\underline{Maggs JQ}, Floros C, Pereira MAM, Schleyer MH. 2010. Rapid visual assessment of fish communities on selected reefs in the Bazaruto Archipelago. \textit{Western Indian Ocean Journal of Marine Science} 9(1): 115-134.\

\underline{Maggs JQ}. 2014. Game fish migrations. In: Goble BJ, van der Elst RP and Oellerman LK (eds). Ugu Lwethu - Our coast: a profile of coastal KwaZulu-Natal. Department of Agriculture and Environmental Affairs and the Oceanographic Research Institute, Cedara, p87.

\underline{Maggs JQ}. 2022. Bottom trawling by the vessel FV Chips in relation to the Napier Port Offshore Disposal Area. NIWA Client Report 2022199AK: 42.

\underline{Maggs JQ}, Mann BQ, Els M, Khumalo D. Mselegu X. 2017. National Marine Linefish System: KZN recreational fisheries monitoring data collection by Ezemvelo KZN Wildlife: 2016 Annual Report. Oceanographic Research Institute, Durban: 60pp. (ORI Unpublished Report 342).

\underline{Maggs JQ},  Mann BQ. 2016. Pondoland Marine Protected Area: Ten years (2006-2016) of monitoring the effectiveness of the Pondoland MPA in protecting offshore reef-fish. Oceanographic Research Institute, Durban: 27 pp. (ORI Unpublished Report 329).

\underline{Maggs JQ}, Dunlop SD, Mann BQ. 2012. National Marine Linefish System: A snapshot evaluation of recreational data collection by Ezemvelo KwaZulu-Natal Wildlife during 2009 and 2010 with recommendations for improvement. Oceanographic Research Institute, Durban: 39pp. (ORI Unpublished Report 296).

\end{hangparas}
\end{document}


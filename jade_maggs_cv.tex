\documentclass[10pt,a4paper]{article}
\usepackage[latin1]{inputenc}
\usepackage{amsmath}
\usepackage{amsfonts}
\usepackage{amssymb}
\usepackage{graphicx}
\usepackage[T1]{fontenc}
\usepackage[left=1.5cm, right=1.5cm, top=1.5cm, bottom=1.5cm]{geometry}
\usepackage[svgnames]{xcolor} %To access named colors below
\usepackage{hyperref}
\hypersetup{colorlinks,citecolor=DeepPink4,linkcolor=DarkRed, urlcolor=DarkBlue}

\author{Jade Maggs}
\title{Jade Maggs Resume}

\begin{document}
\begin{center}
\color{gray}CURRICULUM VITAE\\
\end{center}
\noindent
{\Large \textbf{Jade Quinton Maggs, PhD}}\\
Citizenship: South African\\
{\small Nelson, New Zealand | +6421 056 0873 | \href{mailto:jademaggs@gmail.com}{jademaggs@gmail.com} |
\href{http://www.linkedin.com/pub/jade-maggs/29/917/5a0}{LinkedIn} |
			\href{https://www.researchgate.net/profile/Jade_Maggs}{ResearchGate} | 
			\href{https://www.scopus.com/authid/detail.uri?authorId=55250227800}{Scopus} |
			\href{https://github.com/jademaggs?tab=repositories}{GitHub} |
			\textit{h}-index: 6}\\
\hrule
\vspace{6pt}
\noindent
Fisheries scientist with over a decade's experience. Published research in international peer-reviewed journals. Awarded a PhD in fisheries science from Rhodes University in Grahamstown, South Africa in 2017. Interested in the ecology and biology of marine resources with a focus on applied research. Lectured on stock assessment and R programming at postgraduate level. Strong points are quantitative data analysis, R programming, database administration and written communication.\\
\hrule
\vspace{6pt}
\noindent
EMPLOYMENT HISTORY\\
\\
\begin{tabular}{p{20mm} l }
   Jan 2019 - & \href{https://www.mpi.govt.nz/}{MINISTRY FOR PRIMARY INDUSTRIES} - New Zealand \\
   present & \textbf{Senior Analyst}\\[.2\normalbaselineskip]
   		   & -Participation in Marine Protected Area Science Advisory Group\\[.2\normalbaselineskip]
   		   & -Developing a pathway to ecosystem-based fisheries management\\[.2\normalbaselineskip]
           & -Assessment of the effect of proposed marine protected area network on a rock lobster fishery\\[.2\normalbaselineskip]
           & -Facilitation of customary fisheries processes\\[.2\normalbaselineskip]
           & -Fishery stakeholder engagement (rock lobster fishery)\\
\end{tabular}
\\[.5\normalbaselineskip]
\begin{tabular}{p{20mm}l}
   May 2006 - & \href{https://www.saambr.org.za/research/}{OCEANOGRAPHIC RESEARCH INSTITUTE} - South Africa \\
   Dec 2018   & \textbf{Scientist}: Jun 2018 - Dec 2018\\[.2\normalbaselineskip]
              & \textbf{Assistant Scientist}: Aug 2011 - May 2018\\[.2\normalbaselineskip]
              & \textbf{Senior Scientific Technician}: Jul 2008 - Jul 2011\\[.2\normalbaselineskip]
              & -Monitored inshore recreational fisheries and boat launch site usage\\[.2\normalbaselineskip]
              & -Evaluated marine protected area effectiveness\\[.2\normalbaselineskip]
              & -Investigated fish movement\\[.2\normalbaselineskip]
		      & -Age, growth and stock assessment of commercial teleost species\\[.2\normalbaselineskip]
		      & -Developed and administered fisheries monitoring databases (PostgreSQL)\\[.2\normalbaselineskip]
		      & -Managed fisheries monitoring team\\[.2\normalbaselineskip]
		      & -Participated in Linefish Scientific Working Group\\[.2\normalbaselineskip]
		      & -Collaborated with national and provincial fisheries agencies\\[.2\normalbaselineskip]
\end{tabular}
\\[.5\normalbaselineskip]
\begin{tabular}{p{20mm}l}
   			  & \textbf{Scientific Technician}: May 2006 - Jun 2008\\[.2\normalbaselineskip]
              & -Assessment of fish communities in the Bazaruto Archipelago - consultancy\\[.2\normalbaselineskip]
              & -Management and design of environmental monitoring databases\\[.2\normalbaselineskip]
		      & -Management of a research aquarium\\[.2\normalbaselineskip]
		      & -Developed coral husbandry and propagation facilities\\[.2\normalbaselineskip]
\end{tabular}
\\[.5\normalbaselineskip]
\begin{tabular}{p{20mm}l}
    Jul 2004 - & \href{https://www.saambr.org.za/ushaka-sea-world//}{USHAKA SEA WORLD} - South Africa\\[.2\normalbaselineskip]	
    Apr 2006   & \textbf{Aquarist II}: Feb 2006 - Apr 2006\\[.2\normalbaselineskip]
               & \textbf{Aquarist I}: Feb 2005 - Jan 2006\\[.2\normalbaselineskip]
               & \textbf{Commercial Diver}: Jul 2004 - Jan 2005\\[.2\normalbaselineskip]
	           & -Showed an affinity for animal husbandry - promoted from Diver to Aquarist\\[.2\normalbaselineskip]
		       & -Maintained aquarium exhibits and cultures laboratory\\[.2\normalbaselineskip]
		       & -Developed fish breeding and coral propagation facilities\\[.2\normalbaselineskip]
\end{tabular}
\\
\hrule	
\vspace{6pt}
\noindent
ACADEMIC QUALIFICATIONS
	\begin{itemize}
		\setlength\itemsep{0.05em}
		\item PhD (Fisheries Science) | Rhodes University, 2017
		\begin{itemize}
		\setlength\itemsep{0.05em}
			\item Thesis title: Movement of coastal fishery species in southern Africa: research trends, characterisation of 	behaviours and a case study on fishery implications. Supervisor: Paul Cowley.
		\end{itemize}
		\item MSc (Biology) | University of KwaZulu-Natal, 2011
		\begin{itemize}
		\setlength\itemsep{0.05em}
			\item Thesis title: Fish surveys in exploited and protected areas of the Pondoland Marine Protected Area with consideration of the impact of the MPA on coastal fisheries. Supervisors: Bruce Mann \& Paul Cowley.
		\end{itemize}
		\item BSc (Zoology, specialising in environmental management) | University of South Africa, 2009 
	\end{itemize}
\newpage
\begin{center}
\color{gray}CURRICULUM VITAE\\
\end{center}
\hrule
\vspace{6pt}
\noindent
FUNDING SOURCES (South Africa)
	\begin{itemize}
	\setlength\itemsep{0.05em}
	\item Department of Environmental Affairs, 2011-2018. Project: Monitoring the effectiveness of the Pondoland Marine Protected Area (Project code: 083) - US\$85190
	\item iSimangaliso Wetland Park/Global FinPrint, 2016-2018. Project: Stereo-BRUV survey of the iSimangaliso Wetland Park (Project code: 401) - US\$25000
	\item Ezemvelo KwaZulu-Natal Wildlife, 2015-2016. Project: KwaZulu-Natal Recreational Fisheries Monitoring (Project code: RD75) - US\$28400
	\item National Research Foundation, 2014. Knowledge, Interchange and Collaboration travel grant (UID: 91735) - US\$800
	\item Department of Agriculture, Forestry and Fisheries, 2008-2014. Project: KwaZulu-Natal Recreational Fisheries Monitoring (Project code: RD75) - US\$99300  
	\end{itemize}
\hrule	
\vspace{6pt}
\noindent
LECTURING EXPERIENCE
	\begin{itemize}
		\setlength\itemsep{0.05em}
		\item Fishery stock assessment (University of KwaZulu-Natal, 2012-2018) - Postgraduate
		\item Introduction to R Programming (2016, 2017) - Postgraduate
	\end{itemize}

\hrule
\vspace{6pt}
\noindent
SKILLS
	\begin{itemize}
		\setlength\itemsep{0.05em}
		\item \textbf{Software}: R programming language | PostgreSQL | Ubuntu Linux | 
		ArcGIS Pro | Quantum GIS | Python | \LaTeX\
		\item \textbf{Statistics}: Univariate hypothesis testing | generalised linear modelling | 
		multivariate data analysis
		\item \textbf{Database}: Development of relational SQL databases | PostgreSQL server setup | data consolidation | SQL query writing
		\item \textbf{Communication}: Peer-reviewed journal publication | postgraduate lecturing | oral 
		presentations at conferences | stakeholder engagement
		\item \textbf{Field}: Class IV Commercial Diver (563 hours) | Class IV Commercial Dive Supervisor 
		(313 hours) | Commercial Boat Skipper | Level 3 First Aid | servicing underwater environmental 
		monitoring equipment | fish tagging and fish handling | specimen collection (biological and					environmental)
	\end{itemize}	

\hrule
\vspace{6pt}
\noindent
COURSES ATTENDED
	\begin{itemize}
		\setlength\itemsep{0.05em}
		\item Zonation decision support software training (NIWA, Sep 2019)
		\item Ecofish stock assessment interpretation (Dept. of Agriculture, Forestry and Fisheries, Apr 2017)
		\item Introduction to \LaTeX{} (University of KwaZulu-Natal, Jun 2016)
		\item Introduction to Fisheries Genetics (University of Stellenbosch, Mar 2016)
		\item Species distribution modelling (University of Stellenbosch, Feb 2013) 
		\item Ecological modelling (IRD/University of Cape Town, Jun 2012)
		\item Statistical modelling Using R (Troms\o\ University, Sep 2011)
		\item Fish stock assessment (FAO/SWIOFP - Kenya, Sep-Oct 2010)
	\end{itemize}
\hrule
\vspace{6pt}
\noindent
SHIP TIME
	\begin{itemize}
		\setlength\itemsep{0.05em}
		\item FRV Africana, Natal Bight, Sep 2010 - 5 days - retrieve and deploy ADCPs
		\item TB Davies, Thukela Banks, Apr 2008 - 3 days benthic grab sampling
		\item Ocean Stroom, Thukela Banks, Aug 2005 - 1 day - relocate two bull sharks
		\item Ocean Stroom, Illovo Banks, Aug 2004 - 1 day - specimen collection
	\end{itemize}
\hrule
\vspace{6pt}
\noindent
REFEREES
\\
\\
\textbf{Dr Richard Saunders} | Senior Analyst | Fisheries New Zealand, Nelson\\
Contact: +6422 642 3965 | \href{mailto:richard.saunders@mpi.govt.nz}{richard.saunders@mpi.govt.nz}\\
\\
\noindent
\textbf{Dr Bruce Mann} | Senior Scientist | Oceanographic Research Institute, South Africa\\
Contact: +2772 356 6629 | \href{mailto:bruce@ori.org.za}{bruce@ori.org.za}\\
\\
\noindent
\textbf{Dr Paul de Bruyn} | Science Manager | Indian Ocean Tuna Commission, Seychelles\\
Contact: +248 422 5494 | \href{mailto:paul.debruyn@fao.org}{paul.debruyn@fao.org}\\
\newpage
\begin{center}
\color{gray}CURRICULUM VITAE\\
\end{center}
\hrule
\vspace{6pt}
\noindent
PUBLICATIONS\\
\\
Publication summary: 17 peer-reviewed journal articles, 1 book section, 2 theses, 2 conference proceedings, 70 unpublished reports, 18 oral presentations, 2 posters, 16 popular articles\\ 
\\
\textit{\textbf{Journal}}\\
\noindent
MacNeil MA, Chapman D, Heupel M, [...], \underline{Maggs JQ}. \textit{accepted}. Global status and conservation potential of reef sharks. \textit{Nature}.\\
\\
Dames V, Bernard A, Floros C, Mann B, Speed C, \underline{Maggs J}, Laing S, Meekan M, Olbers J. \textit{In press}. Zonation and reef size significantly influence fish population structure in an established marine protected area, iSimangaliso Wetland Park, South Africa. \textit{Ocean and Coastal Management}.\\
\\
Heyns-Veale ER, Bernard ATF, G\"otz A, Mann BQ, \underline{Maggs JQ}, Smith MKS. 2019. Community-wide effects of protection reveal insights into marine protected area effectiveness for reef fish. \textit{Marine Ecology Progress Series} 620: 99-117.\\
\\
\underline{Maggs JQ}, Cowley PD, Porter SN, Childs AR. 2019. Should I stay or should I go? Intra-population variability in movement behaviour of wide-ranging and resident coastal fishes. \textit{Marine Ecology Progress Series} 619: 111-124.\\
\\ 
Murray TS, Cowley PD, Mann BQ, \underline{Maggs JQ}, Gouws G. 2019. Movement patterns of an endemic South African sparid, the black musselcracker \textit{Cymatoceps nasutus}, determined using mark-recapture methods. \textit{African Journal of Marine Science} 41(1): 71-81.\\
\\  
Mann BQ, Winker H, \underline{Maggs JQ}, Porter S. 2016. Monitoring the recovery of a previously exploited surf-zone fish community in the St Lucia Marine Reserve, South Africa, using a no-take sanctuary area as a benchmark. \textit{African Journal of Marine Science} 38(3): 423-441.\\
\\
\underline{Maggs JQ}, Cowley PD. 2016. Nine decades of fish movement research in southern Africa: a synthesis of research and findings from 1928 to 2014. \textit{Reviews in Fish Biology and Fisheries} 26(3): 287-302.\\
\\
\underline{Maggs JQ}, Mann BQ, Potts WM, Dunlop SW. 2015. Traditional management strategies fail to arrest a decline in the CPUE of an iconic marine recreational fishery species with evidence of hyperstability. \textit{Fisheries Management and Ecology} 23(3-4): 187-199.\\
\\
Dunlop SW, Mann BQ, Cowley PD, Murray TS, \underline{Maggs JQ}. 2015. Movement patterns of \textit{Lichia amia} (Teleostei: Carangidae): results from a long-term cooperative tagging project in South Africa. \textit{African Journal of Zoology} 50(3): 249-257.\\
\\
Mann BQ, \underline{Maggs JQ}, Khumalo MC, Khumalo D, Parak O, Wood J, Bachoo S. 2015. KwaZulu-Natal Boat Launch Site Monitoring System: A novel approach for improved management of small vessels in the coastal zone. \textit{Ocean and Coastal Management} 104(2015): 57-64.\\
\\
Floros C, Schleyer MH, \underline{Maggs JQ}. 2013. Fish as indicators of diving and fishing pressure on high-latitude coral reefs. \textit{Ocean and Coastal Management} 84: 130-139.\\
\\ 
\underline{Maggs JQ}, Mann BQ, Cowley PD. 2013. Reef fish display station-keeping and ranging behaviour in the Pondoland Marine Protected Area on the east coast of South Africa. \textit{African Journal of Marine Science} 35(2): 183-193.\\
\\
\underline{Maggs JQ}, Mann BQ, Cowley PD. 2013. The contribution of a large no-take area to the management of four heavily exploited near-shore linefish species along the south-east African coast. \textit{Fisheries Research} 144(2013): 38-47.\\
\\
Floros C, Schleyer MH, \underline{Maggs JQ}, Celliers L. 2012. Baseline assessment of high-latitude coral reef fish communities in southern Africa. \textit{African Journal of Marine Science} 34(1): 55-69.\\
\\
\underline{Maggs JQ}, Mann BQ, van der Elst RP. 2012. Long-term trends in the management of the recreational shore-fishery for elf (\textit{Pomatomus saltatrix}: Pomatomidae) in KwaZulu-Natal, South Africa. \textit{African Journal of Marine Science} 34(3): 401-410.\\
\\
Mann BQ, Pradervand, P, \underline{Maggs JQ}, Wintner S. 2012. A characterisation of the paddle-ski fishery in KwaZulu-Natal, South Africa. \textit{African Journal of Marine Science} 34(1): 119-130.\\
\\
\underline{Maggs JQ}, Floros C, Pereira MAM, Schleyer MH. 2010. Rapid visual assessment of fish communities on selected reefs in the Bazaruto Archipelago. \textit{Western Indian Ocean Journal of Marine Science} 9(1): 115-134.\\	
\\
\textit{\textbf{Book section}}\\
\underline{Maggs JQ}. 2014. Game fish migrations. In: Goble BJ, van der Elst RP and Oellerman LK (eds). Ugu Lwethu - Our coast: a profile of coastal KwaZulu-Natal. Department of Agriculture and Environmental Affairs and the Oceanographic Research Institute, Cedara, p87.\\
\\
\textit{\textbf{Selected reports}}\\
\underline{Maggs J}, Parker D, Kerwath S, Attwood C, da Silva C, Winker H. 2017. The 2017 assessment of slinger (\textit{Chrysoblephus puniceus}) for the South African linefishery. Report of the Linefish Scientific Working Group, LSWG No. 01, 2017: 27 pp.\\
\\
\underline{Maggs JQ}, Mann BQ, Els M, Khumalo D. Mselegu X. 2017. National Marine Linefish System: KZN recreational fisheries monitoring data collection by Ezemvelo KZN Wildlife: 2016 Annual Report. Oceanographic Research Institute, Durban: 60pp. (ORI Unpublished Report 342).\\
\\
\underline{Maggs JQ} and Mann BQ. 2016. Pondoland Marine Protected Area: Ten years (2006-2016) of monitoring the effectiveness of the Pondoland MPA in protecting offshore reef-fish. Oceanographic Research Institute, Durban: 27 pp. (ORI Unpublished Report 329).\\
\\
\underline{Maggs JQ}, Dunlop SD, Mann BQ. 2012. National Marine Linefish System: A snapshot evaluation of recreational data collection by Ezemvelo KwaZulu-Natal Wildlife during 2009 and 2010 with recommendations for improvement. Oceanographic Research Institute, Durban: 39pp. (ORI Unpublished Report 296).\\
\\
\underline{Maggs JQ}, Schleyer MH, Floros CD, Pereira MAM. 2008. Surveys of reef fish communities relative to recent seismic exploration in the Bazaruto Archipelago on behalf of Sasol in 2007. Oceanographic Research Institute, Durban: 9p. (ORI Unpublished Report 256).\\
\\
\underline{Maggs JQ}, Schleyer MH, Videira E. 2007. Ichthyofauna of the Bazaruto Archipelago: A preliminary study. Oceanographic Research Institute, Durban: 13p including appendices. (ORI Unpublished Report 245).\\
\\
\textit{\textbf{Selected oral presentations}}\\
Freeman D, Ford R, Funnell G, Geange S, \underline{Maggs J}, Nutsford C, Riding T, Sharp B, Tellier P, Tunley KL. 2019. Building an information base for science-based marine protection. New Zealand Marine Sciences Society Conference, 2-5 July 2019, Dunedin, New Zealand. Oral presentation.\\
\\
\underline{Maggs JQ}, Cowley PD, Porter SN, Mann BQ. 2017. From resident to migrant: an empirical classification of coastal fish movement in South Africa. South African Marine Science Symposium, 4-7 July 2017, Boardwalk Convention Centre, Port Elizabeth, South Africa. Oral presentation.\\
\\
\underline{Maggs JQ}, Cowley PD. 2015. Tracking movement of important marine and estuarine fish species in South Africa: a synthesis of research and findings. 9th WIOMSA Scientific Symposium, 26-31 October 2015, Wild Coast Sun Resort, Eastern Cape, South Africa. Oral presentation.\\
\\
\underline{Maggs JQ}, Mann BQ, Potts WM, Dunlop SW. 2014. Long-term declines in CPUE of an iconic recreational species along the South African east coast. 15th Southern African Marine Science Symposium, 15-18 July 2014, Stellenbosch. Oral presentation.\\
\\
\underline{Maggs JQ}, Mann BQ, Cowley PD. 2012. Pondoland Marine Protected Area: indirect effects of protection on the wider fish community. Symposium of Contemporary Conservation Practice, Fern Hill, Pietermaritzburg, 22-26 October 2012. Oral presentation.\\
\\
\underline{Maggs JQ}, Mann BQ. 2012. National Marine Linefish System: Collecting catch and effort data in KwaZulu-Natal. Cape Nature MPAs, Islands and Estuaries Forum, 14 June 2012. Oral presentation.\\ 
\\
\underline{Maggs JQ}, Mann BQ, Cowley PD. 2012. Fish movements in the Pondoland Marine Protected Area: balancing conservation and fisheries enhancement. The 4th Line-fish Symposium, Geelbek, Langebaan, 16-20 April 2012. Oral presentation.\\
\\
\underline{Maggs JQ}, Mann BQ, Cowley PD. 2011. Rebuilding depleted line-fish stocks in the Pondoland Marine Protected Area and adjacent fisheries. The 14th South African Marine Science Symposium, 4-7 April 2011, Rhodes University, Grahamstown. Oral Presentation.
\end{document}


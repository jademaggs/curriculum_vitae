\documentclass[10pt,a4paper]{article}
\usepackage[latin1]{inputenc}
\usepackage{amsmath}
\usepackage{amsfonts}
\usepackage{amssymb}
\usepackage{fancyhdr} % headers and footers
\usepackage{graphicx}
\usepackage[T1]{fontenc}
\usepackage[left=1.5cm, right=1.5cm, top=1.5cm, bottom=1.5cm]{geometry}
\usepackage[svgnames]{xcolor} %To access named colors below
\usepackage{hyperref}
\hypersetup{colorlinks,citecolor=DeepPink4,linkcolor=DarkRed, urlcolor=DarkBlue}
\usepackage{lastpage}

\pagestyle{fancy} % Adds header bar 

\author{Jade Maggs}
\title{Jade Maggs Resume}

% ------------------------------------------------------------------
% Set up headers and footers
% ------------------------------------------------------------------
\fancyhf{} % sets both header and footer to nothing
\renewcommand{\headrulewidth}{0pt}

%\chead{CURRICULUM VITAE}
\lfoot{}
\cfoot{}
\rfoot[\footnotesize{\sffamily{}}]{page \thepage\ of \pageref{LastPage}}

% ------------------------------------------------------------------
% Document start
% ------------------------------------------------------------------
\begin{document}
	
\noindent
{\Large \textbf{Dr Jade Quinton Maggs}}\\
South African citizen with New Zealand permanent residency\\
{\small Auckland, New Zealand | +6421 056 0873 | \href{mailto:jademaggs@gmail.com}{jademaggs@gmail.com} |
\href{http://www.linkedin.com/pub/jade-maggs/29/917/5a0}{LinkedIn} |
			\href{https://www.researchgate.net/profile/Jade_Maggs}{ResearchGate} | 
			\href{https://www.scopus.com/authid/detail.uri?authorId=55250227800}{Scopus \textit{h}-index: 10} |
			\href{https://github.com/jademaggs?tab=repositories}{GitHub}}\\
\hrule
\vspace{6pt}
\noindent
I am a Quantitative Fisheries Scientist with more than 10 years' working experience and a record of internationally collaborative research. Currently, I am working for the National Institute of Water and Atmospheric Research (NIWA) in Auckland with a focus on surveys to inform stock assessment. I manage projects that use creel surveys to collect catch and effort data in the New Zealand recreational boat-based fishery, primarily for inshore fin-fish and lobster. In 2017, I was awarded a PhD in fisheries science from Rhodes University in South Africa. I also have experience with data management and GIS, and I have lectured part-time at postgraduate level on fishery stock assessment and R programming. I am a highly organised individual and I excel at leading change. \\
%Previously, I held a Senior Analyst role with Fisheries New Zealand within the Fisheries Management department. Before that I held a Scientist role with the Oceanographic Research Institute in South Africa with a focus on monitoring marine recreational fisheries and marine protected areas. I have published quantitative fisheries research in international peer-reviewed journals. In 2017, I was awarded a PhD in fisheries science from Rhodes University in South Africa. I have also lectured part-time at postgraduate level on fishery stock assessment and R programming. I am a highly organised individual and I excel at leading change. 
\hrule
\vspace{6pt}
\noindent

% ------------------------------------------------------------------
% EMPLOYMENT HISTORY
% ------------------------------------------------------------------

\noindent
EMPLOYMENT HISTORY\\
\\
\begin{tabular}{p{20mm} l }
Apr 2020 - 	& \href{https://www.niwa.co.nz/}{NATIONAL INSTITUTE OF WATER AND ATMOSPHERIC RESEARCH (NIWA)} - New Zealand \\[.3\normalbaselineskip]
present	& \textbf{Group Manager - Modelling \&Recreational Fisheries}\\[.2\normalbaselineskip]
			& - Leading recreational boat fishery surveys (creel, camera)\\[.2\normalbaselineskip]		
			& - Fisheries Modelling (GLM, CPUE standardisation, harvest estimation)\\[.2\normalbaselineskip]	
			& - Focus on inshore fin-fish and rock lobster fisheries\\[.2\normalbaselineskip]	
			& - Fishery characterisation\\[.2\normalbaselineskip]	
			& - Working group participation (Marine Amateur Fisheries, Inshore Fisheries Assessment)\\[.2\normalbaselineskip]	
			& - Leading team of three staff (two scientists, one technician) and also project-specific teams\\[.2\normalbaselineskip]	
			& - Using R Programming Language and QGIS\\[.2\normalbaselineskip]	
			& - Leading implementation of electronic data acquisition and streamlining workflows\\[.2\normalbaselineskip]	
			& - Selected for in-house Leadership Development Programme\\[.5\normalbaselineskip]	
		& \textbf{Group Manager - Fisheries Data Services}\\[.2\normalbaselineskip]
	           	& - Managed NIWA and Ministry for Primary Industries fisheries data (research and observer)\\[.2\normalbaselineskip]
	           	& - Managed commercial rock lobster catch sampling project\\[.2\normalbaselineskip]
   	           	& - Led development of data loading software and PostgreSQL databases\\[.2\normalbaselineskip]
	           	& - Managed a team of six staff\\[.2\normalbaselineskip]
			& - Used R programming language, PostgreSQL, Linux, QGIS, GitLab\\[.2\normalbaselineskip]
			& - Established efficiencies and notable improvements in multiple workflows\\

\end{tabular}
\\[.5\normalbaselineskip]
\begin{tabular}{p{20mm} l }
Jan 2019 - 	& \href{https://www.mpi.govt.nz/}{MINISTRY FOR PRIMARY INDUSTRIES (MPI)} - New Zealand \\
Mar 2020   	& \textbf{Senior Analyst (Fisheries Management, Fisheries New Zealand)}\\[.2\normalbaselineskip]
			& - Conceptualised a pathway to ecosystem-based fisheries management\\[.2\normalbaselineskip]
     			& - Participated in Marine Protected Area (MPA) Science Advisory Group\\[.2\normalbaselineskip]
			& - Fishery stakeholder engagement (MPA network proposed near commercial rock lobster fishery)\\
\end{tabular}
\\[.5\normalbaselineskip]
\begin{tabular}{p{20mm}l}
May 2006 - 	& \href{https://www.saambr.org.za/research/}{OCEANOGRAPHIC RESEARCH INSTITUTE} - South Africa \\
Dec 2018   	& \textbf{Scientist}: Jun 2018 - Dec 2018\\[.2\normalbaselineskip]
 		& \textbf{Assistant Scientist}: Aug 2011 - May 2018\\[.2\normalbaselineskip]
		& \textbf{Senior Scientific Technician}: Jul 2008 - Jul 2011\\[.2\normalbaselineskip]
 			& - Led recreational fisheries surveys (shore and boat) under government contract\\[.2\normalbaselineskip]
			& - Fisheries modelling (GLM, CPUE standardisation, fish movement)\\[.2\normalbaselineskip]
	            	& - Monitored relative abundance, movement and community composition of fishes in the Pondoland MPA\\[.2\normalbaselineskip]
         			& - Postgraduate lecturing on stock assessment and R programming\\[.2\normalbaselineskip]
			& -  Biological sampling of commercial fish species\\[.2\normalbaselineskip]
		      	& - Developed and administered fisheries monitoring databases (PostgreSQL)\\[.2\normalbaselineskip]
		      	& - Working Group participation (Linefish Scientific Working Group)\\[.2\normalbaselineskip]
              		& - Team management (three staff)\\[.2\normalbaselineskip]
			& - Report writing\\
\end{tabular}
\\[.5\normalbaselineskip]
\begin{tabular}{p{20mm}l}
	& \textbf{Scientific Technician}: May 2006 - Jun 2008\\[.2\normalbaselineskip]
		& - Assessment of fish communities in the Bazaruto Archipelago - consultancy\\[.2\normalbaselineskip]
		& - Managed and designed environmental monitoring databases\\[.2\normalbaselineskip]
		& - Managed a research aquarium laboratory\\[.2\normalbaselineskip]
		& - Developed coral husbandry and propagation facilities\\
\end{tabular}
\\[.5\normalbaselineskip]
\begin{tabular}{p{20mm}l}
Jul 2004 - 	& \href{https://www.saambr.org.za/ushaka-sea-world//}{USHAKA SEA WORLD} - South Africa\\[.2\normalbaselineskip]	
Apr 2006   	& \textbf{Aquarist II}: Feb 2006 - Apr 2006\\[.2\normalbaselineskip]
               	& \textbf{Aquarist I}: Feb 2005 - Jan 2006\\[.2\normalbaselineskip]
               	& \textbf{Commercial Diver}: Jul 2004 - Jan 2005\\[.2\normalbaselineskip]
	           	& - Showed an affinity for animal husbandry - promoted from Diver to Aquarist\\[.2\normalbaselineskip]
		       	& - Maintained aquarium exhibits and cultures laboratory\\[.2\normalbaselineskip]
		      	& - Developed fish breeding and coral propagation facilities\\[.2\normalbaselineskip]
\end{tabular}
\\
\hrule	
\vspace{6pt}
\noindent

% ------------------------------------------------------------------
% CURRENT PROJECTS
% ------------------------------------------------------------------

\noindent
CURRENT PROJECTS
	\begin{itemize}
	\setlength\itemsep{0.05em}
	\item Design for a longline survey to estimate the age structure of New Zealand h\a=apuku
	\item Web camera monitoring of key marine amateur fisher access points
	\item Monitoring of recreational harvest of rock lobsters in CRA 2
	\item Monitoring the Kaik\a=oura and Motunau boat-based marine amateur fisheries using a bus route survey
	\item Mortality rates of fish released by recreational fishers
	\item Estimation of recreational harvest in shellfish fisheries
	\item National mean fish weight survey
	\end{itemize}
\hrule	
\vspace{6pt}

% ------------------------------------------------------------------
% ACADEMIC QUALIFICATIONS
% ------------------------------------------------------------------

\noindent
ACADEMIC QUALIFICATIONS
	\begin{itemize}
		\setlength\itemsep{0.05em}
		\item PhD (Fisheries Science) | Rhodes University, 2017
		\begin{itemize}
		\setlength\itemsep{0.05em}
			\item Thesis title: Movement of coastal fishery species in southern Africa: research trends, characterisation of behaviours and a case study on fishery implications. Supervisor: Prof. Paul Cowley.
		\end{itemize}
		\item MSc (Biology) | University of KwaZulu-Natal, 2011
		\begin{itemize}
		\setlength\itemsep{0.05em}
			\item Thesis title: Fish surveys in exploited and protected areas of the Pondoland Marine Protected Area with consideration of the impact of the MPA on coastal fisheries. Supervisors: Dr Bruce Mann \& Prof. Paul Cowley.
		\end{itemize}
		\item BSc (Zoology, specialising in environmental management) | University of South Africa, 2009 
	\end{itemize}

\hrule
\vspace{6pt}

\noindent
LECTURING EXPERIENCE
	\begin{itemize}
		\setlength\itemsep{0.05em}
		\item Fishery stock assessment (University of KwaZulu-Natal, 2012-2018) - Postgraduate
		\item Introduction to R Programming (2016, 2017) - Postgraduate
	\end{itemize}

\hrule
\vspace{6pt}
\noindent
TECHNICAL SKILLS
	\begin{itemize}
		\setlength\itemsep{0.05em}
		\item \textbf{Software}: R programming | GIT | Quantum GIS | Ubuntu Linux | Bash | \LaTeX\
		\item \textbf{Statistics}: Maximum likelihood | Generalised linear modelling
		\item \textbf{Data management}: PostgreSQL 
		%\item \textbf{Communication}: Peer-reviewed journal publication | postgraduate lecturing | oral presentations at conferences
		\item \textbf{Field}: Class IV Commercial Diver (563 hours) | Class IV Commercial Dive Supervisor 
		(313 hours) | Commercial Boat Skipper | Level 2 First Aid | STCW - PST, PSSR | servicing underwater environmental 
		monitoring equipment | fish handling and tagging | specimen collection (biological/environmental)
	\end{itemize}	

\hrule
\vspace{6pt}
\noindent
MODELLING COURSES ATTENDED
	\begin{itemize}
		\setlength\itemsep{0.05em}
		\item Zonation decision support software training (NIWA, Sep 2019)
		\item Ecofish stock assessment interpretation (Dept. of Agriculture, Forestry and Fisheries, Apr 2017)
		\item Introduction to \LaTeX{} (University of KwaZulu-Natal, Jun 2016)
		\item Introduction to Fisheries Genetics (University of Stellenbosch, Mar 2016)
		\item Species distribution modelling (University of Stellenbosch, Feb 2013) 
		\item Ecological modelling (IRD/University of Cape Town, Jun 2012)
		\item Statistical modelling Using R (Troms\o\ University, Sep 2011)
		\item Fish stock assessment (FAO/SWIOFP - Kenya, Sep-Oct 2010)
	\end{itemize}
\hrule
\vspace{6pt}
\noindent
SHIP TIME
	\begin{itemize}
		\setlength\itemsep{0.05em}
		\item RV Kaharoa, Cook Strait, Jul/Aug 2021 - 2 days - KAH2105 acoustic snapshot, trawling
		\item FRV Africana, Natal Bight, Sep 2010 - 5 days - retrieve and deploy ADCPs
		\item TB Davies, Thukela Banks, Apr 2008 - 3 days benthic grab sampling
		\item Ocean Stroom, Thukela Banks, Aug 2005 - 1 day - relocate two bull sharks
		\item Ocean Stroom, Illovo Banks, Aug 2004 - 1 day - specimen collection
	\end{itemize}
\hrule
\vspace{6pt}
\noindent
REFEREES
\\
\\
\textbf{Mr Jonathan Moores} | Regional Manager | NIWA, Auckland (current manager)\\
Contact:+64 27 233 7871 | \href{mailto:Jonathan.Moores@niwa.co.nz}{Jonathan.Moores@niwa.co.nz}\\
\\
\textbf{Dr Alison MacDiarmid} | Regional Manager | NIWA, Wellington\\
Contact: +64 274 033 468 | \href{mailto:Alison.MacDiarmid@niwa.co.nz}{Alison.MacDiarmid@niwa.co.nz}\\
\\
\textbf{Dr Richard Saunders} | Impact \& Engagement Officer | IMOS, Tasmania\\
Contact: +61 483 350 831 | \href{mailto:Richard.Saunders@utas.edu.au}{Richard.Saunders@utas.edu.au}\\
\\
%\noindent
%\textbf{Dr Bruce Mann} | Senior Scientist | Oceanographic Research Institute, South Africa\\
%Contact: +2772 356 6629 | \href{mailto:bruce@ori.org.za}{bruce@ori.org.za}\\
%\\
\noindent
\textbf{Dr Paul de Bruyn} | Science Manager | Indian Ocean Tuna Commission, Seychelles\\
%Contact: +248 422 5494 | \href{mailto:paul.debruyn@fao.org}{paul.debruyn@fao.org}\\
Contact: +34 671 57 31 13 | \href{mailto:paul.debruyn@fao.org}{paul.debruyn@fao.org}\\

\hrule
\newpage
\vspace{6pt}
\noindent
SELECTED PUBLICATIONS\\
\\
Publication summary: 19 peer-reviewed journal articles, 1 book section, 2 theses, 2 conference proceedings, 76 unpublished reports, 18 oral presentations, 2 posters, 16 popular articles\\ 
\\
\textit{\textbf{Journal}}\\
\noindent
\underline{Maggs JQ}, Hartill BW, Evans O, Holdsworth J, Lumley T. in prep. Anatomical hooking location and capture depth lower survival probability in snapper \textit{Chrysophrys auratus} caught by recreational fishers. \textit{Fisheries Research}.\\ 
\\
Goetze J, Heithaus MR, MacNeil MA, [...], \underline{Maggs JQ}, [...]. in prep. Management diversity is the best approach for the conservation of reef sharks.\\
\\
Simpfendorfer CA, Heithaus MR, Heupel MR, [...], \underline{Maggs JQ}, [...], Chapman D. 2023. Widespread diversity deficits of coral reef sharks and rays. \textit{Science} 380: 1155-1160.\\
\\
Mann BQ, \underline{Maggs JQ}, Porter SN, Dalton WN. 2022. Monitoring the effects of spatial protection on the reef fish communities of the Pondoland Marine Protected Area, Eastern Cape Province, South Africa. \textit{African Journal of Marine Science} 44(2): 1-17.\\
\\
Dames V, Bernard ATF, Floros C, Mann B, Speed C, \underline{Maggs J}, Laing S, Meekan M, Olbers J. 2020. Zonation and reef size significantly influence fish population structure in an established marine protected area, iSimangaliso Wetland Park, South Africa. \textit{Ocean \& Coastal Management} 185: 105040.\\
\\
MacNeil MA, Chapman D, Heupel M, [...], \underline{Maggs JQ}. 2020. Global status and conservation potential of reef sharks. \textit{Nature} 583: 801-806.\\
\\
Dames V, Bernard A, Floros C, Mann B, Speed C, \underline{Maggs J}, Laing S, Meekan M, Olbers J. 2020. Zonation and reef size significantly influence fish population structure in an established marine protected area, iSimangaliso Wetland Park, South Africa. \textit{Ocean and Coastal Management} 185:105040.\\
\\
Heyns-Veale ER, Bernard ATF, G\"otz A, Mann BQ, \underline{Maggs JQ}, Smith MKS. 2019. Community-wide effects of protection reveal insights into marine protected area effectiveness for reef fish. \textit{Marine Ecology Progress Series} 620: 99-117.\\
\\
\underline{Maggs JQ}, Cowley PD, Porter SN, Childs AR. 2019. Should I stay or should I go? Intra-population variability in movement behaviour of wide-ranging and resident coastal fishes. \textit{Marine Ecology Progress Series} 619: 111-124.\\
\\ 
Murray TS, Cowley PD, Mann BQ, \underline{Maggs JQ}, Gouws G. 2019. Movement patterns of an endemic South African sparid, the black musselcracker \textit{Cymatoceps nasutus}, determined using mark-recapture methods. \textit{African Journal of Marine Science} 41(1): 71-81.\\
\\  
Mann BQ, Winker H, \underline{Maggs JQ}, Porter S. 2016. Monitoring the recovery of a previously exploited surf-zone fish community in the St Lucia Marine Reserve, South Africa, using a no-take sanctuary area as a benchmark. \textit{African Journal of Marine Science} 38(3): 423-441.\\
\\
\underline{Maggs JQ}, Cowley PD. 2016. Nine decades of fish movement research in southern Africa: a synthesis of research and findings from 1928 to 2014. \textit{Reviews in Fish Biology and Fisheries} 26(3): 287-302.\\
\\
\underline{Maggs JQ}, Mann BQ, Potts WM, Dunlop SW. 2015. Traditional management strategies fail to arrest a decline in the CPUE of an iconic marine recreational fishery species with evidence of hyperstability. \textit{Fisheries Management and Ecology} 23(3-4): 187-199.\\
\\
Dunlop SW, Mann BQ, Cowley PD, Murray TS, \underline{Maggs JQ}. 2015. Movement patterns of \textit{Lichia amia} (Teleostei: Carangidae): results from a long-term cooperative tagging project in South Africa. \textit{African Journal of Zoology} 50(3): 249-257.\\
\\
Mann BQ, \underline{Maggs JQ}, Khumalo MC, Khumalo D, Parak O, Wood J, Bachoo S. 2015. KwaZulu-Natal Boat Launch Site Monitoring System: A novel approach for improved management of small vessels in the coastal zone. \textit{Ocean and Coastal Management} 104(2015): 57-64.\\
\\
Floros C, Schleyer MH, \underline{Maggs JQ}. 2013. Fish as indicators of diving and fishing pressure on high-latitude coral reefs. \textit{Ocean and Coastal Management} 84: 130-139.\\
\\ 
\underline{Maggs JQ}, Mann BQ, Cowley PD. 2013. Reef fish display station-keeping and ranging behaviour in the Pondoland Marine Protected Area on the east coast of South Africa. \textit{African Journal of Marine Science} 35(2): 183-193.\\
\\
\underline{Maggs JQ}, Mann BQ, Cowley PD. 2013. The contribution of a large no-take area to the management of four heavily exploited near-shore linefish species along the south-east African coast. \textit{Fisheries Research} 144(2013): 38-47.\\
\\
Floros C, Schleyer MH, \underline{Maggs JQ}, Celliers L. 2012. Baseline assessment of high-latitude coral reef fish communities in southern Africa. \textit{African Journal of Marine Science} 34(1): 55-69.\\
\\
\underline{Maggs JQ}, Mann BQ, van der Elst RP. 2012. Long-term trends in the management of the recreational shore-fishery for elf (\textit{Pomatomus saltatrix}: Pomatomidae) in KwaZulu-Natal, South Africa. \textit{African Journal of Marine Science} 34(3): 401-410.\\
\\
Mann BQ, Pradervand, P, \underline{Maggs JQ}, Wintner S. 2012. A characterisation of the paddle-ski fishery in KwaZulu-Natal, South Africa. \textit{African Journal of Marine Science} 34(1): 119-130.\\
\\
\underline{Maggs JQ}, Floros C, Pereira MAM, Schleyer MH. 2010. Rapid visual assessment of fish communities on selected reefs in the Bazaruto Archipelago. \textit{Western Indian Ocean Journal of Marine Science} 9(1): 115-134.\\	
\\
\textit{\textbf{Book section}}\\
\underline{Maggs JQ}. 2014. Game fish migrations. In: Goble BJ, van der Elst RP and Oellerman LK (eds). Ugu Lwethu - Our coast: a profile of coastal KwaZulu-Natal. Department of Agriculture and Environmental Affairs and the Oceanographic Research Institute, Cedara, p87.\\
\\
\textit{\textbf{Selected reports}}\\
\underline{Maggs JQ}, Hartill BW, Evans O, Holdsworth J, Lumley T, Stevens T. accepted. Mortality rates of snapper released by recreational fishers. \textit{New Zealand Fisheries Assessment Report 2023/xx}. xx p\\
\\
\underline{Maggs JQ}, Parsons D. 2023. Design for catch sampling programme to estimate the age structure of New Zealand h\a=apuku (\textit{Polyprion oxygeneios}). \textit{New Zealand Fisheries Assessment Report 2023/34}. 30 p\\
\\
\underline{Maggs JQ}, Davey NK, Hartill BW. 2023. Monitoring the Kaik\=oura and Motunau boat-based marine amateur fisheries for blue cod, sea perch and rock lobster, 2021-22. \textit{New Zealand Fisheries Assessment Report 2023/33}. 53 p\\
\\
\underline{Maggs JQ}. 2022. Bottom trawling by the vessel FV Chips in relation to the Napier Port Offshore Disposal Area. NIWA Client Report 2022199AK: 42.\\
\\
Schnabel KE, Mills VS, Tracey DM, Macpherson D, Kelly M, Peart RA, \underline{Maggs J}, Yeoman J, Wood CR. 2021. Identification of benthic invertebrate samples from research trawls and observer trips, 2020?21. Ministry for Primary Industries, Wellington: 51 pp (New Zealand Aquatic Environment and Biodiversity Report No. 269).\\
\\
\underline{Maggs J}, Parker D, Kerwath S, Attwood C, da Silva C, Winker H. 2017. The 2017 assessment of slinger (\textit{Chrysoblephus puniceus}) for the South African linefishery. Report of the Linefish Scientific Working Group, LSWG No. 01, 2017: 27 pp.\\
\\
\underline{Maggs JQ}, Mann BQ, Els M, Khumalo D. Mselegu X. 2017. National Marine Linefish System: KZN recreational fisheries monitoring data collection by Ezemvelo KZN Wildlife: 2016 Annual Report. Oceanographic Research Institute, Durban: 60pp. (ORI Unpublished Report 342).\\
\\
\underline{Maggs JQ} and Mann BQ. 2016. Pondoland Marine Protected Area: Ten years (2006-2016) of monitoring the effectiveness of the Pondoland MPA in protecting offshore reef-fish. Oceanographic Research Institute, Durban: 27 pp. (ORI Unpublished Report 329).\\
\\
\underline{Maggs JQ}, Dunlop SD, Mann BQ. 2012. National Marine Linefish System: A snapshot evaluation of recreational data collection by Ezemvelo KwaZulu-Natal Wildlife during 2009 and 2010 with recommendations for improvement. Oceanographic Research Institute, Durban: 39pp. (ORI Unpublished Report 296).\\
\\
%\underline{Maggs JQ}, Schleyer MH, Floros CD, Pereira MAM. 2008. Surveys of reef fish communities relative to recent seismic exploration in the Bazaruto %Archipelago on behalf of Sasol in 2007. Oceanographic Research Institute, Durban: 9p. (ORI Unpublished Report 256).\\
%\\
%\underline{Maggs JQ}, Schleyer MH, Videira E. 2007. Ichthyofauna of the Bazaruto Archipelago: A preliminary study. Oceanographic Research Institute, Durban: %13p including appendices. (ORI Unpublished Report 245).\\
%\\
\textit{\textbf{Selected oral presentations}}\\
Freeman D, Ford R, Funnell G, Geange S, \underline{Maggs J}, Nutsford C, Riding T, Sharp B, Tellier P, Tunley KL. 2019. Building an information base for science-based marine protection. New Zealand Marine Sciences Society Conference, 2-5 July 2019, Dunedin, New Zealand. Oral presentation.\\
\\
\underline{Maggs JQ}, Cowley PD, Porter SN, Mann BQ. 2017. From resident to migrant: an empirical classification of coastal fish movement in South Africa. South African Marine Science Symposium, 4-7 July 2017, Boardwalk Convention Centre, Port Elizabeth, South Africa. Oral presentation.\\
\\
\underline{Maggs JQ}, Cowley PD. 2015. Tracking movement of important marine and estuarine fish species in South Africa: a synthesis of research and findings. 9th WIOMSA Scientific Symposium, 26-31 October 2015, Wild Coast Sun Resort, Eastern Cape, South Africa. Oral presentation.\\
\\
\underline{Maggs JQ}, Mann BQ, Potts WM, Dunlop SW. 2014. Long-term declines in CPUE of an iconic recreational species along the South African east coast. 15th Southern African Marine Science Symposium, 15-18 July 2014, Stellenbosch. Oral presentation.\\
\\
\underline{Maggs JQ}, Mann BQ, Cowley PD. 2012. Pondoland Marine Protected Area: indirect effects of protection on the wider fish community. Symposium of Contemporary Conservation Practice, Fern Hill, Pietermaritzburg, 22-26 October 2012. Oral presentation.\\
\\
\underline{Maggs JQ}, Mann BQ. 2012. National Marine Linefish System: Collecting catch and effort data in KwaZulu-Natal. Cape Nature MPAs, Islands and Estuaries Forum, 14 June 2012. Oral presentation.\\ 
\\
\underline{Maggs JQ}, Mann BQ, Cowley PD. 2012. Fish movements in the Pondoland Marine Protected Area: balancing conservation and fisheries enhancement. The 4th Line-fish Symposium, Geelbek, Langebaan, 16-20 April 2012. Oral presentation.\\
\\
\underline{Maggs JQ}, Mann BQ, Cowley PD. 2011. Rebuilding depleted line-fish stocks in the Pondoland Marine Protected Area and adjacent fisheries. The 14th South African Marine Science Symposium, 4-7 April 2011, Rhodes University, Grahamstown. Oral Presentation.
\end{document}

